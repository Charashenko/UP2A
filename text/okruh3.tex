\section{Identifikace, autentizace a datové schránky}

\subsection{Pojem elektronického podpisu a současné legislativní změny; rozdíl oproti elektronické pečeti.}
\begin{itemize}
      \item  Elektronická formá právního jednání
            \begin{enumerate}
                  \item Písemnost
                        \begin{itemize}
                              \item Elektronické prostředky nenahrazují právní jednání v písemné formě, jsou jejich jiným
                                    projevem stojícím paralelně vedle něj – jsou rovnocenné
                        \end{itemize}
                  \item Podpis jednajícího - cokoli co indetifukuje subjekt a je připojeno k dalším datům
                        \begin{itemize}
                              \item Podpis = Podpis / virtuální identita – emailová adresa, avatár, uživatelský účet, IP adresa, el.
                                    podpis, platnost od 2000, IDENTIFIKACE A INTEGRITA DOKUMENTU
                              \item Oblast el. identifikace – certifikace, ověřování, zabezpečení, spolupráce států (Amerika X
                                    Evropa)
                              \item stavěn na úroveň klasickému podpisu
                              \item Nařízení eIDAS: \uv{data v elektronické podobě, která jsou připojena k jiným datům v elektronické podobě nebo
                                          jsou s nimi logicky spojena a která podepisující osoba používá k podepsání}
                              \item Druhy: prostý, zaručený, kvalifikovaný el. podpis (certifikát)
                              \item dokument podepisuje veřejný orgán, vždy nutnost podepsat kvalifikovaným el. podpisem
                              \item  V případě podepisování dokumentu soukromou osobou v případě komunikace s veřejným
                                    orgánem – nutnost kvalifikovaného elektronického podpisu
                              \item Mimo výkon veřejné moci jakýkoliv podpis


                        \end{itemize}
            \end{enumerate}
      \item Občanský zákoník (elektronická kontraktace
      \item Nařízení Evropského parlamentu a Rady (EU) č. 910/2014 ze dne 23. července 2014 o elektronické identifikaci
            a službách vytvářejících důvěru pro elektronické transakce na vnitřním trhu a o zrušení směrnice 1999/93/ES
            (eIDAS)
      \item Zákon o elektronické identifikaci
      \item Zákon o elektronických úkonech a autorizované konverzi dokumentů
      \item Zákon o archivnictví a spisové službě
      \item  Elektronická pečeť slouží jako důkaz toho, že elektronický dokument vydala určitá právnická osoba, a poskytuje jistotu o původu a integritě tohoto dokumentu. \textbf{Není spojena s konkrétní osobou.}
\end{itemize}


\subsection{Nařízení eIDAS -- důvody přijetí}
\begin{itemize}
      \item eIDAS - Electronic indetification and services
      \item Nařízení Evropského parlamentu a Rady (EU) č. 910/2014 ze dne 23. července 2014 o elektronické identifikaci
            a službách vytvářejících důvěru pro elektronické transakce na vnitřním trhu a o zrušení směrnice 1999/93/ES
      \item Důvody:
            \begin{itemize}
                  \item  Dotvoření digitálního volného vnitřního trhu
                  \item Zvýšení důveryhodnosti elektronických transakcí
                  \item Vytvoření jednotného rámce pro elektronickou identifikaci
                  \item Prokazování totožnosti v rámci EU
                  \item Nahrazení současné legislativy pro elektronické podpisy
            \end{itemize}
\end{itemize}


\subsection{Charakterizujte a popište právní otázky související s elektornickou identifikací ve smyslu prokázání totožnosti.}


\subsection{Datové schránky - shrňte výhody, nevýhody a mj. se zaměřte na okamžik doručení.}
\item Co to je ?\begin{itemize}
    \begin{itemize}
        \item Podle nařízení eIDAS čl. 3 bod 38 služba, která umožňuje přenášet data mezi třetími osobami elektronickými prostředky a poskytuje důkazy týkající se nakládání s přenášenými daty včetně dokladu o odeslání a přijetí dat, které chrání přenášená data před rizikem ztráty, krádeže, nebo poškození
        \item Spolehlivý způsob komunikace s veřejnou správou 
        \item Fikce doručení - garantuje doručení
        \item Spravováno Ministerstvem vnitra
    \end{itemize}
\item\textbf{Operace datových schránek}
    \begin{itemize}
        \item Odeslat
        \item Přijmout
        \item Ověření stavu odeslané zprávy
        \item Přijmou doklad o dodání a doručení
        \item Ověření, zda adresát má datovou schránku
    \end{itemize}
\item\textbf{Fikce doručení}
    \begin{itemize}
        \item Lhůta 10 dní - Po uplinutí této lhůty je zpráva považována za přečtenou
        \item Fakticky se jedná o povinnost pravidelně kontrolovat datovou schránku
        \item Účelem je zefektivnění procesů veřejné správy
    \end{itemize}
\item \textbf{Výhody:} el. komunikace se státními orgány, dostupnost, úspora času, nižší náklady, možnost zplnomocnění
\item \textbf{Nevýhody:} Nutnost pravidelné kontroly, omezení na 1 schránka/firma nezávisle na velikosti společnosti.
\end{itemize}


\subsection{Judikatura k elektronické identifikaci subjektu a datovým schránkám.}
\begin{itemize}
    \item eIDAS č.3 ods. 1- Elektronická identifikace subjektu- postup pou
    //TODO zítra
\end{itemize}
